\chapter{数据结构}

% \lstinputlisting[language={C++}]{./src/ch02/}
%---------------------------------------
\section{花式打表}\small
%---------------------------------------


%---------------------------------------
	\subsection{并查集}\small
%---------------------------------------
%------------------------------------------------------------------------------%
维护一些不相交的集合, 支持两种操作: 合并两个集合, 查询一个元素所处的集合.

维护一个森林, 每一棵树表示一个集合, 树根元素为这个集合的代表元. 数组
{\tt father[]} 维护每一个元素的父节点. 在查询时, 只需要不断寻找父节点, 即可以找
到该元素所处集合的代表元. 在合并时, 先找到这两个集合的代表元 x, y, 然后令
{\tt father[x] := y} 即可. 优化 1: 路径压缩, 在沿着树根的路径找到某一元素 a 的所
在集合的代表元 b 时, 将路径上的所有元素 x 直接进行 {\tt father[x] := b} 操作. 优
化 2: 启发式合并, 用数组 {\tt rank[]} 维护集合的高度, 每次将较小的集合合并到较大
的集合上, 而合并两个相同高度的集合时更新高度.

\begin{longtable}{|c|l|l|}
成员变量 & & \\
& \verb`int fa[]` & 元素的父节点 \\
& \verb`int rk[]` & 集合的高度, rank \\
成员函数 & & \\
$O(n)$ & \verb`void init(int n)` & 初始化并查集, n 个独立元素 \\
$O(1)$ & \verb`int find(int x)` & 查找 x 所在集合的代表元 \\
$O(1)$ & \verb`bool query(int x, int y)` & x, y 是否在同一集合 \\
$O(1)$ & \verb`void merge(int x, int y)` & 合并 x, y 两个集合 \\
\end{longtable}
\lstinputlisting[language={C++}]{./src/ch02/0201-DisjointSet.cpp}


%---------------------------------------
	\subsection{块状链表}\small
%---------------------------------------


%---------------------------------------
	\subsection{ST表}\small
%---------------------------------------


%---------------------------------------
\section{平衡树}\small
%---------------------------------------


%---------------------------------------
	\subsection{红黑树 RBTree}\small
%---------------------------------------


%---------------------------------------
	\subsection{树堆 Treap}\small
%---------------------------------------


%---------------------------------------
	\subsection{伸展树 Splay}\small
%---------------------------------------


%---------------------------------------
\section{二叉堆}\small
%---------------------------------------


%---------------------------------------
	\subsection{优先队列}\small
%---------------------------------------


%---------------------------------------
	\subsection{左偏树}\small
%---------------------------------------


%---------------------------------------
\section{线段树}\small
%---------------------------------------


%---------------------------------------
	\subsection{树状数组}\small
%---------------------------------------


%---------------------------------------
	\subsection{RMQ 线段树}\small
%---------------------------------------


%---------------------------------------
\section{树链剖分}\small
%---------------------------------------


%---------------------------------------
\section{动态树}\small
%---------------------------------------


%---------------------------------------
\section{手指树}\small
%---------------------------------------


%---------------------------------------
\section{Trie}\small
%---------------------------------------

