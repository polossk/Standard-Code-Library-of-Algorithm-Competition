\chapter{其他}
%---------------------------------------
\section{自用常见函数及定义}\small
%---------------------------------------
\lstinputlisting[language={C++}]{./src/ch99/01.cpp}


%---------------------------------------
\section{输入输出外挂}\small
%---------------------------------------
用于加速整数的输入输出,主要应对较大的数据量.
\lstinputlisting[language={C++}]{./src/ch99/02.cpp}


%---------------------------------------
\section{高精度}\small
%---------------------------------------
一般最好使用Java,因为可以省去大量的码代码的时间.不到万不得已不要用完全
大数模板.


%---------------------------------------
    \subsection{大数模板}\small
%---------------------------------------
大数模板,根据需要添加函数.
\lstinputlisting[language={C++}]{./src/ch99/03.cpp}


%---------------------------------------
    \subsection{Java BigInteger}\small
%---------------------------------------
利用Java来处理大数问题是一件很轻松愉快的事情,不仅写起来简单方便,而且能
够省去大量的调试时间.不过运算函数最好全部写成函数名调用,尽量不要直接使
用运算符调用.

数据定义与输入输出
{\tt
    \begin{longtable}{|p{9.3cm}|p{6cm}|}
    \hline
    方法 & 作用 \\
    \hline
    BigInteger n = new BigInteger(val) & \\
    \hline
    BigInteger n = new BigInteger(val, radix) & radix表示进制 \\
    \hline
    Scanner cin = Scanner(System.in); & 读入一个大数 \\
    n = cin.nextBigInteger(); & \\
    \hline
    System.out.print(n); & 输出大数n \\
    \hline
    System.out.println(n); & 输出大数n并且换行 \\
    \hline
    System.out.printf("\%d\textbackslash n", n); & 类似C++格式化输出大数n \\
    \hline
    \end{longtable}
}

基本常量与方法
{\tt
    \begin{longtable}{|p{9.3cm}|p{6cm}|}
    \hline
    方法 & 作用 \\
    \hline
    BigInteger.ONE & 常量,值为1 \\
    \hline
    BigInteger.TEN & 常量,值为10 \\
    \hline
    BigInteger.ZERO & 常量,值为0 \\
    \hline
    String toString() & 返回10进制下的字符串 \\
    \hline
    String toString(int radix) & 返回基于radix进制下的字符串 \\
    \hline
    BigInteger valueOf(int val) & 将val的值赋给this \\
    \hline
    int compareTo(BigInteger val) & 判断大小并返回-1, 0, 1 \\
    \hline
    boolean equals(Object x) & 判断是否与指定的Object相等 \\
    \hline
    \end{longtable}
}

运算方法
{\tt
    \begin{longtable}{|p{9.3cm}|p{6cm}|}
    \hline
    方法 & 作用 \\
    \hline
    BigInteger abs() & 返回其绝对值 \\
    \hline
    BigInteger negate() & 返回其相反数 \\
    \hline
    int signum() & 返回其符号函数 \\
    \hline
    BigInteger add(BigInteger val) & 返回 this + val \\
    \hline
    BigInteger subtract(BigInteger val) & 返回 this - val \\
    \hline
    BigInteger muliply(BigInteger val) & 返回 this * val \\
    \hline
    BigInteger divide(BigInteger val) & 返回 this / val \\
    \hline
    BigInteger remainder(BigInteger val) & 返回 this \% val \\
    \hline
    BigInteger[] divideAndRemainder(val) & a[] = this / val, this \% val \\
    \hline
    BigInteger mod(BigInteger val) & 返回 this mod val \\
    \hline
    BigInteger gcd(BigInteger val) & 返回 gcd(this, val) \\
    \hline
    BigInteger pow(BigInteger val) & 返回 this \^{} val \\
    \hline
    BigInteger max(BigInteger val) & 返回两者的最大值 \\
    \hline
    BigInteger min(BigInteger val) & 返回两者的最小值 \\
    \hline
    BigInteger and(BigInteger val) & 返回 this and val \\
    \hline
    BigInteger andNot(BigInteger val) & 返回 this and \~{}val \\
    \hline
    BigInteger not(BigInteger val) & 返回 this not val \\
    \hline
    BigInteger or(BigInteger val) & 返回 this or val \\
    \hline
    BigInteger xor(BigInteger val) & 返回 this xor val \\
    \hline
    BigInteger shiftLeft(int n) & 返回 this << n \\
    \hline
    BigInteger shiftRight(int n) & 返回 this >> n \\
    \hline
    \end{longtable}
}

%---------------------------------------
    \subsection{Java BigDecimal}\small
%---------------------------------------
BigDecimal 的基本用法与 BigInteger 大同小异,所以关于其基本方法不再赘述.
在此给出两个常用的方法:
{\tt
    \begin{longtable}{|p{9.3cm}|p{6cm}|}
    \hline
    方法 & 作用 \\
    \hline
    BigDecimal stripTrailingZeros() & 去除后导零 \\
    \hline
    String toPlainString() & 返回非科学计数法环境下的数值 \\
    \hline
    \end{longtable}
}


%---------------------------------------
\section{分数类}\small
%---------------------------------------
{\it{ 警告: 该代码的运行速度很慢, 请谨慎使用.}}

完成分数的加减乘除运算.成员变量为分子(num)与分母(den),
只能通过分子分母来进行构造,
并且重载了+, -, *, /, < ,==运算符.
\lstinputlisting[language={C++}]{./src/ch99/04.cpp}


%---------------------------------------
\section{进制转换}\small
%---------------------------------------
把一个x进制的数转换成y进制.

具体做法是先把x进制转换为10进制,然后在不断地取模倒序转换成y进制.

若高进制的字母表示有区别,请注意同时修改两处字母.
\begin{longtable}{|c|l|l|}
复杂度 & $O(L)$ &  \\
输入 & x & 原数据进制数,$2 \leq x \leq 62$ \\
 & y & 新数据进制数, $2 \leq x \leq 62$ \\
 & s & 原数据的字符串形式 \\
输出 & string & 新数据的字符串形式 \\
\end{longtable}
\lstinputlisting[language={C++}]{./src/ch99/05.cpp}


%---------------------------------------
\section{格雷码}\small
%---------------------------------------
给一个二进制的位数n,求出一个0到$2^n-1$的排列,使得相邻两项(包括首尾相邻)
的二进制表达中只有恰好一位不同.

由数学知识可知,一种简单的格雷码编码方式有规律如下:$g[i] = i \text{xor} (i >> 1)$

\begin{longtable}{|c|l|l|}
复杂度 & $O(2^{n})$ & \\
输入 & n & 二进制的位数 \\
输出 & vector<int> & n位的格雷码序列 \\
\end{longtable}
\lstinputlisting[language={C++}]{./src/ch99/06.cpp}