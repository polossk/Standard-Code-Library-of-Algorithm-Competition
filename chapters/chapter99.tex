\clearpage\chapter{其他}

%---------------------------------------
\section{输入输出外挂}\small
%---------------------------------------
用于加速整数的输入输出,主要应对较大的数据量.
\lstinputlisting[language={C++}]{./src/ch99/01.cpp}


%---------------------------------------
\section{高精度}\small
%---------------------------------------
一般最好使用Java,因为可以省去大量的码代码的时间.不到万不得已不要用完全
大数模板.


%---------------------------------------
    \subsection{大数模板}\small
%---------------------------------------
大数模板,根据需要添加函数.
\lstinputlisting[language={C++}]{./src/ch99/02.cpp}


%---------------------------------------
    \subsection{Java BigInteger}\small
%---------------------------------------
利用Java来处理大数问题是一件很轻松愉快的事情,不仅写起来简单方便,而且能
够省去大量的调试时间.不过运算函数最好全部写成函数名调用,尽量不要直接使
用运算符调用.

数据定义与输入输出
\begin{longtable}{|p{6.2cm}|p{2.7cm}|p{6.2cm}|}
\hline
方法 & 接收参数 & 作用 \\
\hline
 BigInteger n = new BigInteger(val) & String, int, void & 新建一个值为val的大数 \\
\hline
 BigInteger n = new BigInteger(val, radix) & String, int & 新建一个在radix进制下的值为val的大数 \\
\hline
 Scanner cin = Scanner(System.in); & & 读入一个大数 \\
 n = cin.nextBigInteger(); & & \\
\hline
 System.out.print(n); & & 输出大数n \\
\hline
 System.out.println(n); & & 输出大数n并且换行 \\
\hline
 System.out.printf("\%d$\backslash$n", n); & & 类似C++格式化输出大数n \\
\hline
\end{longtable}

基本常量与方法 
\begin{longtable}{|p{6.2cm}|p{2.7cm}|p{6.2cm}|}
\hline
方法 & 接收参数 & 作用 \\
\hline
 BigInteger.ONE & & 常量,值为1 \\
\hline
 BigInteger.TEN & & 常量,值为10 \\
\hline
 BigInteger.ZERO & & 常量,值为0 \\
\hline
 String toString() & void & 返回10进制下的字符串表示形式 \\
\hline
 String toString(radix) & int & 返回基于radix进制下的字符串表示形式 \\
\hline
 BigInteger valueOf(val) & long, int & 将val的值赋给this \\
\hline
 int compareTo(val) & BigInteger & 根据小于、等于或大于val返回-1, 0, 1 \\
\hline
 boolean equals(x) & Object & 判断是否与指定的Object相等 \\
\hline
\end{longtable}

运算方法
\begin{longtable}{|p{6.2cm}|p{2.7cm}|p{6.2cm}|}
\hline
 方法 & 接收参数 & 作用 \\
\hline
 BigInteger abs() & void & 返回其绝对值 \\
\hline
 BigInteger negate() & void & 返回其相反数 \\
\hline
 int signum() & void & 返回其符号函数 \\
\hline
 BigInteger add(val) & BigInteger & 返回一值为(this + val)的大数 \\
\hline
 BigInteger subtract(val) & BigInteger & 返回一值为(this - val)的大数 \\
\hline
 BigInteger muliply(val) & BigInteger & 返回一值为(this * val)的大数 \\
\hline
 BigInteger divide(val) & BigInteger & 返回一值为(this / val)的大数 \\
\hline
 BigInteger remainder(val) & BigInteger & 返回一值为(this \% val)的大数 \\
\hline
 BigInteger[] divideAndRemainder(val) & BigInteger & a[0] = this / val \\
 & & a[1] = this \% val \\
\hline
 BigInteger mod(val) & BigInteger & 返回一值为(this mod val)的大数 \\
\hline
 BigInteger gcd(val) & BigInteger & 返回一值为gcd(this, val)的大数 \\
\hline
 BigInteger pow(val) & BigInteger & 返回一值为(this \^ val)的大数 \\
\hline
 BigInteger max(val) & BigInteger & 返回this, val的最大值 \\
\hline
 BigInteger min(val) & BigInteger & 返回this, val的最小值 \\
\hline
 BigInteger and(val) & BigInteger & 返回一值为(this and val)的大数 \\
\hline
 BigInteger andNot(val) & BigInteger & 返回一值为(this and ~val)的大数 \\
\hline
 BigInteger not(val) & BigInteger & 返回一值为(this not val)的大数 \\
\hline
 BigInteger or(val) & BigInteger & 返回一值为(this or val)的大数 \\
\hline
 BigInteger xor(val) & BigInteger & 返回一值为(this xor val)的大数 \\
\hline
 BigInteger shiftLeft(n) & int & 返回一值为(this << n)的大数 \\
\hline
 BigInteger shiftRight(n) & int & 返回一值为(this >> n)的大数 \\
\hline
\end{longtable}


%---------------------------------------
    \subsection{Java BigDecimal}\small
%---------------------------------------
BigDecimal 的基本用法与 BigInteger 大同小异,所以关于其基本方法不再赘述.
在此给出两个常用的方法:
\begin{longtable}{|p{6.2cm}|p{2.7cm}|p{6.2cm}|}
\hline
方法 & 接收参数 & 作用 \\
\hline
 BigInteger stripTrailingZeros() & void & 去除后导零 \\
\hline
 String toPlainString() & void & 返回非科学计数法环境下的数值 \\
\hline
\end{longtable}


%---------------------------------------
\section{分数类}\small
%---------------------------------------
完成分数的加减乘除运算.成员变量为分子(num)与分母(den),
只能通过分子分母来进行构造,
并且重载了+, -, *, /, < ,==运算符.
\lstinputlisting[language={C++}]{./src/ch99/03.cpp}


%---------------------------------------
\section{进制转换}\small
%---------------------------------------
把一个x进制的数转换成y进制.

具体做法是先把x进制转换为10进制,然后在不断地取模倒序转换成y进制.

若高进制的字母表示有区别,请注意同时修改两处字母.
\begin{longtable}{|c|l|l|}
复杂度 & $O(L)$ &  \\
输入 & x & 原数据进制数,$2 \leq x \leq 62$ \\
 & y & 新数据进制数, $2 \leq x \leq 62$ \\
 & s & 原数据的字符串形式 \\
输出 & string & 新数据的字符串形式 \\ 
\end{longtable}
\lstinputlisting[language={C++}]{./src/ch99/04.cpp}


%---------------------------------------
\section{格雷码}\small
%---------------------------------------
给一个二进制的位数n,求出一个0到$2^n-1$的排列,使得相邻两项(包括首尾相邻)
的二进制表达中只有恰好以为不同.

由数学知识可知,一种简单的格雷码编码方式有规律如下:$g[i] = i xor (i >> 1)$

\begin{longtable}{|c|l|l|}
复杂度 & $O(2^{n})$ &  \\
输入  & n & 二进制的位数 \\
输出 & vector<int> & n位的格雷码序列 \\ 
\end{longtable}
\lstinputlisting[language={C++}]{./src/ch99/05.cpp}